\documentclass[12pt,a4paper]{article}
\renewcommand{\baselinestretch}{1.5}
\usepackage{setspace} 
\usepackage[utf8]{inputenc} % til at skrive æÆøØåÅ.
\usepackage{amsmath,amssymb,amsfonts,mathrsfs,latexsym}
\usepackage[british]{babel}
\usepackage{longtable}
\usepackage{ulem}
\usepackage{booktabs}
\usepackage{graphicx}
\usepackage{array}
\usepackage{caption}
\usepackage{subcaption}
\usepackage{titlepic}
\usepackage{enumerate}
\usepackage{enumitem}
\usepackage{float}
\usepackage{pdfpages}
\usepackage[colorlinks = true,
            linkcolor = blue,
            urlcolor  = blue,
            citecolor = blue,
            anchorcolor = blue]{hyperref}
\usepackage[margin=1.0in]{geometry}
\usepackage{framed,color}
\usepackage[explicit]{titlesec}
\usepackage{pgfplots}
\usepgfplotslibrary{fillbetween}
\usepackage{graphicx}
\usepackage{subcaption}
\usepackage[amsmath,thmmarks]{ntheorem} %pakke til at lave sætningsenvorinmets (kan ikke loades sammen med amsthm)
\usepackage{color}

%opsætning af graf-standarder
\setlength{\belowcaptionskip}{20pt plus 3pt minus 2pt}
\pgfplotsset{direct/.append style={axis lines = left,
    xlabel = $p$,
    ylabel = {$x$},
every axis x label/.style={
    at={(ticklabel* cs:1.05)},
    anchor=west,
},
every axis y label/.style={
    at={(ticklabel* cs:1.05)},
    anchor=south,
}    
    }}
\usepackage{caption} 
\captionsetup[table]{skip=15pt}   
\pgfplotsset{inverse/.append style={
	axis lines = left,
    xlabel = $x$,
    ylabel = {$p$},
every axis x label/.style={
    at={(ticklabel* cs:1.05)},
    anchor=west,
},
every axis y label/.style={
    at={(ticklabel* cs:1.05)},
    anchor=south,
}
    }}


%opretter environmets til sætningsstrukturen 
\theorembodyfont{\normalfont}

	
	%sætnings environment	
	\newtheorem{thm}{Theorem}

	\theoremstyle{break}	
	%opgave environment	
	\newtheorem{opg}{Opgave}	

	%Korrolar environment
	\newtheorem{coro}[thm]{Corollary}	
	
	%Lemma environment	
	\newtheorem{lemma}[thm]{Lemma}
	
	\theoremsymbol{\ensuremath{\circ}}	
	
	%definition environment	
	\newtheorem{definition}[thm]{Definition}
	
	%eksempel environment	
	\newtheorem{exam}[thm]{Example}
	
	
	
	%Bevis environment
	\theoremstyle{nonumberplain}
	\theoremheaderfont{%
	\normalfont\itshape}
	\theorembodyfont{\normalfont}
	\theoremsymbol{\ensuremath{\square}}
	\theoremseparator{.}
	
	\newtheorem{proof}{Proof}
		\theoremsymbol{}
	\newtheorem{sol}{Løsningsforslag}
	
\usepackage{natbib}
\usepackage{footmisc}
\usepackage{multirow}
\usepackage{csquotes} % quotes
\usepackage{tabulary}

%\setlength\parindent{0pt}

%\titleformat{\section}{\Large\bfseries}{}{0pt}{#1}
%\titleformat{\subsection}{\large\bfseries}{}{0pt}{#1}


%nye komandoer
\newcommand{\mR}{\mathbb{R}}
\newcommand{\mZ}{\mathbb{Z}}
\newcommand{\mN}{\mathbb{N}}
\newcommand{\mQ}{\mathbb{Q}}
\newcommand{\mC}{\mathbb{C}}
\newcommand{\hs}{\hspace{2mm}}
\newcommand{\Hs}{\hspace{4mm}}
\newcommand{\pipe}{\hs | \hs}
\newcommand{\lp}{\left(}
\newcommand{\rp}{\right)}
\newcommand{\vect}[1]{\underline{#1}}
\newcommand{\matr}[1]{\underline{\underline{#1}}}
\newcommand{\cR}{\hat{R}}
\newcommand{\cM}{\hat{M}}
\newcommand{\gm}{\mathfrak{m}}
\newcommand{\gn}{\mathfrak{n}}
\newcommand{\gR}{\text{gr }R}
\newcommand{\bigslant}[2]{{\raisebox{.2em}{$#1$}\left/\raisebox{-.2em}{$#2$}\right.}}
\DeclareMathOperator{\Ima}{Im}
\newcommand{\squeezeup}{\vspace{-2.5mm}}

%\numberwithin{equation}{section}



%\titlepic{\includegraphics[width=0.7\textwidth]{logo.jpg}}



\begin{document}
\section*{Remote Sensing Change Detection}
The remote sensing analysis was actually done quite simply (one I figured out how to) using QGIS. I will go through each of the steps of the analysis and the accompanying files in the folder.

\begin{enumerate}
\item First I used the SA-TIED data (your data) to detect known mining areas. I did this in Google Earth Enginge.
	\begin{enumerate}
	\item Load in the datapoints from the datafile. 
	\item add a raster layer (the newest with respect to the given time-series you are looking at), in order to actually see the mines on a satellite image.

	\item From here I actually just zoomed in on the mines and manually draw polygons around them, giving each of the a name and lastly saving them all in one shapefile.\\
	Earth Engine script: \url{hhttps://code.earthengine.google.com/?scriptPath=users%2Fannabeckthelin%2Ftest%3ACreate%20mine-polygons}\\
	Output folder: "01 Mine Polygons - Bojanala Platinum District"

	\item Lastly, to visually see if the mine changed whithin the polygons over time, I created a new script with a raster-layer for each year I was considering and added the polygons on top. This was just a visual "test" and not something I used for the analysis.\\
	Earth Engine script: \url{https://code.earthengine.google.com/?scriptPath=users%2Fannabeckthelin%2Ftest%3ALandsat8%20Mine%20Bands}
	\end{enumerate}
\item Next I created the albedo calculated satellite images.
	\begin{enumerate}
	\item I downloaded the relevant top of atmosphere (TOA) reflectance corrected satellite images via United States Geological Surveys (USGS) EarthExplorer (descriptions in thesis).
	\item The output from USGS is a folder containing each of the bands from the given image.\\
	Output folder (one example): "02 USGS TOA corrected images (2014)""
	\item I uploaded the relevant rasters (band 2,3 and 4) into QGIS and simply used the raster calculater to calculate the difuse visible albedo layers (equation in thesis).\\
	Output file: "03 alb. 2014 image.tif"
	\end{enumerate}
\item As a last step in data process, I used R to extract and evaluate the pixelvalues within the mine-polygons.\\
R script: "04 R_extract_pixel_values.R"
\end{enumerate}



\end{document}